% Options for packages loaded elsewhere
\PassOptionsToPackage{unicode}{hyperref}
\PassOptionsToPackage{hyphens}{url}
%
\documentclass[
]{article}
\usepackage{amsmath,amssymb}
\usepackage{lmodern}
\usepackage{ifxetex,ifluatex}
\ifnum 0\ifxetex 1\fi\ifluatex 1\fi=0 % if pdftex
  \usepackage[T1]{fontenc}
  \usepackage[utf8]{inputenc}
  \usepackage{textcomp} % provide euro and other symbols
\else % if luatex or xetex
  \usepackage{unicode-math}
  \defaultfontfeatures{Scale=MatchLowercase}
  \defaultfontfeatures[\rmfamily]{Ligatures=TeX,Scale=1}
\fi
% Use upquote if available, for straight quotes in verbatim environments
\IfFileExists{upquote.sty}{\usepackage{upquote}}{}
\IfFileExists{microtype.sty}{% use microtype if available
  \usepackage[]{microtype}
  \UseMicrotypeSet[protrusion]{basicmath} % disable protrusion for tt fonts
}{}
\makeatletter
\@ifundefined{KOMAClassName}{% if non-KOMA class
  \IfFileExists{parskip.sty}{%
    \usepackage{parskip}
  }{% else
    \setlength{\parindent}{0pt}
    \setlength{\parskip}{6pt plus 2pt minus 1pt}}
}{% if KOMA class
  \KOMAoptions{parskip=half}}
\makeatother
\usepackage{xcolor}
\IfFileExists{xurl.sty}{\usepackage{xurl}}{} % add URL line breaks if available
\IfFileExists{bookmark.sty}{\usepackage{bookmark}}{\usepackage{hyperref}}
\hypersetup{
  pdftitle={Visit Scotland project documentation},
  hidelinks,
  pdfcreator={LaTeX via pandoc}}
\urlstyle{same} % disable monospaced font for URLs
\usepackage[margin=1in]{geometry}
\usepackage{graphicx}
\makeatletter
\def\maxwidth{\ifdim\Gin@nat@width>\linewidth\linewidth\else\Gin@nat@width\fi}
\def\maxheight{\ifdim\Gin@nat@height>\textheight\textheight\else\Gin@nat@height\fi}
\makeatother
% Scale images if necessary, so that they will not overflow the page
% margins by default, and it is still possible to overwrite the defaults
% using explicit options in \includegraphics[width, height, ...]{}
\setkeys{Gin}{width=\maxwidth,height=\maxheight,keepaspectratio}
% Set default figure placement to htbp
\makeatletter
\def\fps@figure{htbp}
\makeatother
\setlength{\emergencystretch}{3em} % prevent overfull lines
\providecommand{\tightlist}{%
  \setlength{\itemsep}{0pt}\setlength{\parskip}{0pt}}
\setcounter{secnumdepth}{-\maxdimen} % remove section numbering
\ifluatex
  \usepackage{selnolig}  % disable illegal ligatures
\fi

\title{Visit Scotland project documentation}
\author{}
\date{\vspace{-2.5em}}

\begin{document}
\maketitle

{
\setcounter{tocdepth}{2}
\tableofcontents
}
\hypertarget{brief-description-of-project-topic}{%
\subsection{Brief description of project
topic}\label{brief-description-of-project-topic}}

\hypertarget{project-brief}{%
\paragraph{Project brief}\label{project-brief}}

Provide insights into Scottish Domestic Tourism for the following based
on 2013 to 2019 Scot Gov data. Day Visits based on: - Activities

\begin{itemize}
\item
  Demographics
\item
  Location Type
\item
  Transportation
\end{itemize}

Regional Tourism based on:

\begin{itemize}
\item
  Areas visited
\item
  Tourist's residency
\end{itemize}

By analysing the data available I am looking to provide Visit Scotland
with key insights into what drives domestic tourism in Scotland and what
areas they can focus on to help maximise tourist spending and visits.

\hypertarget{overview-of-visit-scotland}{%
\paragraph{Overview of Visit
Scotland}\label{overview-of-visit-scotland}}

VisitScotland.com is the official consumer website of VisitScotland,
Scotland's national tourist board. Working closely with private
businesses, public agencies and local authorities, we work to ensure
that our visitors experience the very best of Scotland and that the
country makes the most of its outstanding tourism assets and realises
its potential.

To do this, VisitScotland: • markets Scotland to all parts of the world
to attract visitors. • provides information and inspiration to visitors
and potential visitors so they get the best out of a visit to Scotland.
• provides quality assurance to visitors and quality advice to our
industry partners to help the industry meet - and strive to exceed -
visitors' expectations.

\hypertarget{data}{%
\subsection{Data}\label{data}}

\hypertarget{data-sources}{%
\paragraph{Data sources}\label{data-sources}}

All of the following data was sourced from the scot gov statistics site
.

Regional Domestic Tourism Scottish Accomodation Occupancy Tourism Day
Visits - Activities Tourism Day Visits - Demographics Tourism Day Visits
- Location Tourism Day Visits - Transport

\hypertarget{types-of-data}{%
\paragraph{Types of data}\label{types-of-data}}

The data provided was in the following forms: Categorical - characters
Numeric - doubles

\hypertarget{data-formats}{%
\paragraph{Data formats}\label{data-formats}}

The data come was all downloaded in flat files (CSV) format.

\hypertarget{data-quality-and-bias}{%
\paragraph{Data quality and bias}\label{data-quality-and-bias}}

Looking at the About tab for the datasets on Scottish Gov Survey data,
the data quality control measures for the data included regular quality
control checks at each stage of the survey. These included checks for
the sampling design, selection of sampling methods, design of the
questionnaire, how the survey was delivered, how the data was collected,
entered and cleaned. The survey dataset may be biased because of the
small sample size for some of the categories that the data was gathered
for.

\hypertarget{ethics}{%
\subsection{Ethics}\label{ethics}}

\hypertarget{ethical-issues-in-data-sourcing-and-extraction}{%
\paragraph{Ethical issues in data sourcing and
extraction}\label{ethical-issues-in-data-sourcing-and-extraction}}

There are no obvious ethical considerations, because the data does not
identify particular individuals and their health details.

\hypertarget{ethical-implications-of-business-requirements}{%
\paragraph{Ethical implications of business
requirements}\label{ethical-implications-of-business-requirements}}

The datasets are covered by the Open Government License, which means the
data can be used for private analysis by members of the public, but
should not be reproduced for commercial gain.

\hypertarget{analysis}{%
\subsection{Analysis}\label{analysis}}

\hypertarget{stages-in-the-data-analysis-process}{%
\paragraph{Stages in the data analysis
process}\label{stages-in-the-data-analysis-process}}

What were the main stages in your data analysis process?

\hypertarget{tools-for-data-analysis}{%
\paragraph{Tools for data analysis}\label{tools-for-data-analysis}}

What were the main tools you used for your analysis?

\hypertarget{descriptive-diagnostic-predictive-and-prescriptive-analysis}{%
\paragraph{Descriptive, diagnostic, predictive and prescriptive
analysis}\label{descriptive-diagnostic-predictive-and-prescriptive-analysis}}

Please report under which of the below categories your analysis falls
\textbf{and why} (can be more than one)

\textbf{Descriptive Analytics} tells you what happened in the past.

\textbf{Diagnostic Analytics} helps you understand why something
happened in the past.

\textbf{Predictive Analytics} predicts what is most likely to happen in
the future.

\textbf{Prescriptive Analytics} recommends actions you can take to
affect those outcomes.

\end{document}
